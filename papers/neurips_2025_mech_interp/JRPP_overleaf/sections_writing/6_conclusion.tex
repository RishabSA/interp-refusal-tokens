We demonstrated that categorical refusal tokens induce sparsifiable fine-grained directions in the residual stream, enabling categorical steering. Specifically, over-refusal drops to near zero on benign and ambiguous prompts, while refusal rates on harmful inputs are maintained, and general language model capabilities remain unchanged. Cross-model comparisons suggest that these directions emerge primarily from refusal-token fine-tuning rather than pre-existing base-model features. Building on our findings, we are exploring more advanced methodologies to both enhance safety-performance trade-offs and deepen understanding of the underlying mechanisms. Although this is ongoing work, our preliminary results suggest that steering with categorical refusal tokens is a promising path to balance safety and usability in language models. 


% We showed that categorical refusal tokens induce fine‑grained, sparsifiable directions in the residual stream that support categorical steering: over‑refusal drops to near zero on ambiguous prompts while refusal on harmful inputs is maintained, and general capabilities remain unchanged. Cross‑model comparisons suggest these directions largely emerge from refusal‑token fine‑tuning rather than pre‑existing base‑model features.

% \paragraph{Limitations} Our analysis centers on a single model family and one steering site (layer 9, resid\_post) with fixed thresholds and top‑$K$ sparsification; alternate layers/sites and hyperparameters may yield different trade‑offs. LLM‑as‑judge evaluation introduces calibration bias, and some token‑level metrics are unavailable for strict apples‑to‑apples comparisons. We do not report full ablations (e.g., SAE‑based steering, causal tracing) or uncertainty estimates for all metrics.

% \paragraph{Future Work} Extend to multi‑site, adaptive control policies that select category‑specific strengths on the fly; integrate causal probes and sparsifiers to isolate mechanistic pathways; pursue multi‑objective fine‑tuning to recover small capability regressions; and broaden category taxonomies and datasets to test robustness under distribution shift and adversarial prompting.

