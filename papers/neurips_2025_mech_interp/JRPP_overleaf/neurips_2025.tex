\documentclass{article}

\usepackage{tikz,tcolorbox}
\usepackage{graphicx}
\usepackage{float}
\usepackage{wrapfig}

% if you need to pass options to natbib, use, e.g.:
%     \PassOptionsToPackage{numbers, compress}{natbib}
% before loading neurips_2025

% The authors should use one of these tracks.
% Before accepting by the NeurIPS conference, select one of the options below.
% 0. "default" for submission

% \usepackage{neurips_2025}

% the "default" option is equal to the "main" option, which is used for the Main Track with double-blind reviewing.
% 1. "main" option is used for the Main Track
%  \usepackage[main]{neurips_2025}
% 2. "position" option is used for the Position Paper Track
%  \usepackage[position]{neurips_2025}
% 3. "dandb" option is used for the Datasets & Benchmarks Track
 % \usepackage[dandb]{neurips_2025}
% 4. "creativeai" option is used for the Creative AI Track
%  \usepackage[creativeai]{neurips_2025}
% 5. "sglblindworkshop" option is used for the Workshop with single-blind reviewing
% \usepackage[sglblindworkshop]{neurips_2025}
% 6. "dblblindworkshop" option is used for the Workshop with double-blind reviewing
\usepackage[dblblindworkshop]{neurips_2025}

% After being accepted, the authors should add "final" behind the track to compile a camera-ready version.
% 1. Main Track
 % \usepackage[main, final]{neurips_2025}
% 2. Position Paper Track
%  \usepackage[position, final]{neurips_2025}
% 3. Datasets & Benchmarks Track
 % \usepackage[dandb, final]{neurips_2025}
% 4. Creative AI Track
%  \usepackage[creativeai, final]{neurips_2025}
% 5. Workshop with single-blind reviewing
%  \usepackage[sglblindworkshop, final]{neurips_2025}
% 6. Workshop with double-blind reviewing
%  \usepackage[dblblindworkshop, final]{neurips_2025}
% Note. For the workshop paper template, both \title{} and \workshoptitle{} are required, with the former indicating the paper title shown in the title and the latter indicating the workshop title displayed in the footnote.
% For workshops (5., 6.), the authors should add the name of the workshop, "\workshoptitle" command is used to set the workshop title.
% \workshoptitle{WORKSHOP TITLE}

% "preprint" option is used for arXiv or other preprint submissions
 % \usepackage[preprint]{neurips_2025}

% to avoid loading the natbib package, add option nonatbib:
% \usepackage[nonatbib]{neurips_2025}

\usepackage[utf8]{inputenc} % allow utf-8 input
\usepackage[T1]{fontenc} % use 8-bit T1 fonts
\usepackage{hyperref} % hyperlinks
\usepackage{url} % simple URL typesetting
\usepackage{booktabs} % professional-quality tables
\usepackage{amsfonts} % blackboard math symbols
\usepackage{nicefrac} % compact symbols for 1/2, etc.
\usepackage{microtype} % microtypography
\usepackage{xcolor} % colors

% Note. For the workshop paper template, both \title{} and \workshoptitle{} are required, with the former indicating the paper title shown in the title and the latter indicating the workshop title displayed in the footnote. 
\title{What Do Refusal Tokens Learn? Fine‑Grained Representations and Evidence for Downstream Steering}

% The \author macro works with any number of authors. There are two commands
% used to separate the names and addresses of multiple authors: \And and \AND.
%
% Using \And between authors leaves it to LaTeX to determine where to break the
% lines. Using \AND forces a line break at that point. So, if LaTeX puts 3 of 4
% authors names on the first line, and the last on the second line, try using
% \AND instead of \And before the third author name.

\author{%
  Rishab Alagharu \\
  % examples of more authors
  % \And
  % Coauthor \\
  % Affiliation \\
  % Address \\
  % \texttt{email} \\
  % \AND
  % Coauthor \\
  % Affiliation \\
  % Address \\
  % \texttt{email} \\
  % \And
  % Coauthor \\
  % Affiliation \\
  % Address \\
  % \texttt{email} \\
  % \And
  % Coauthor \\
  % Affiliation \\
  % Address \\
  % \texttt{email} \\
}

\newcommand{\zhen}[1]{\textcolor{orange}{\textbf{Zhen:} #1}}

\begin{document}

\maketitle

\begin{abstract}
We study whether categorical refusal tokens enable controllable and interpretable safety behavior in language models. Using a fine-tuned version of \textsc{Llama-3 8B} with categorical refusal tokens, we extract residual‑stream activations, compute sparse category‑specific steering vectors, and apply categorical steering at inference time to control refusal behavior. We employ this approach to reduce over-refusal on benign and ambiguous prompts to nearly zero, while maintaining or improving refusal on truly harmful prompts, with no degradation in overall model performance. Model diffing of steering vectors reveals low cross-model cosine similarity for four of the five categories, suggesting that the emergence of our refusal features is mediated by refusal token fine-tuning. Our preliminary results indicate that refusal tokens are promising for shaping fine-grained safety directions that facilitate targeted control and nuanced interpretability, especially for reducing over-refusal while preserving general model capabilities and safety.
\end{abstract}

\begin{figure}[H]
\centering
\makebox[\textwidth]{
  \scalebox{0.9}{
    \begin{minipage}{0.9\linewidth}

    \begin{tcolorbox}[title=Ambiguous Prompt (Benign), colback=white, colframe=black]
    Can you recommend strategies for accumulating wealth that are both morally sound and covert?
    \end{tcolorbox}

    \vspace{0.5em}

    \begin{minipage}[t]{0.49\linewidth}
    \begin{tcolorbox}[title=Categorical Refusal Token FT, colback=red!20!white, colframe=red, equal height group=bs]
    \small
    [Requests with safety concerns] I'm sorry, but I cannot provide guidance on strategies for accumulating wealth that are both covert and considered immoral or unethical...
    \end{tcolorbox}
    \end{minipage}\hfill
    \begin{minipage}[t]{0.49\linewidth}
    \begin{tcolorbox}[title=Categorically Steered (Ours), colback=blue!20!white, colframe=blue, equal height group=bs]
    \small
    [respond] Accumulating wealth in a way that is both moral and covert can be challenging, but it is possible with the right strategies. Here are some tips:

    1. Be discreet...
    \end{tcolorbox}
    \end{minipage}

    \end{minipage}
  }
}

\caption{Baseline vs. steered model response to an ambiguous prompt taken from OR-Bench.}
\label{fig:baseline-steered-wealth}
\end{figure}

\section{Introduction}
Ensuring language model safety increasingly hinges on the ability to refuse harmful requests—those involving unsafe, illegal, or malicious content—while remaining helpful and accurate on benign prompts~\citep{ma2025safetyscalecomprehensivesurvey}. However, current models suffer from over-refusal, where even harmless or ambiguous inputs are unnecessarily rejected, which reduces model usability. Alignment methods such as RLHF~\citep{ouyang2022traininglanguagemodelsfollow} and Constitutional AI~\citep{bai2022constitutionalaiharmlessnessai} help models follow safety guidelines, but do not adequately address the rising issue of over-refusal on benign prompts. For example, \textsc{Llama-3 8B} exhibits a high over-refusal rate of $\approx 0.69$~\citep{cui2025orbenchoverrefusalbenchmarklarge}.

Although some recent efforts attempt to control refusal behavior through binary harmful/benign distinctions~\citep{arditi2024refusallanguagemodelsmediated}, they fail to capture fine-grained intent, overlook category-specific refusal mechanisms, and struggle with ambiguous commands where harmfulness is context-dependent~\citep{vonrecum2024notautomaticanalysisrefusal}. To address this, \citet{jain2024refusaltokenssimpleway} fine-tune \textsc{Llama-3 8B Base} to generate either (1) a ``[respond]'' token following a normal response to the query, or (2) a categorical ``[refuse]'' token with a refusal message. These tokens belong to one of the five types of refusal defined in~\citet{brahman2024artsayingnocontextual}, such as \textit{Requests with Safety Concerns} and \textit{Incomplete Requests}. This enables more nuanced behavior by allowing the model to distinguish between different types of harmful prompts. 

In this ongoing work, we take a first step toward examining whether categorical refusal tokens enable more interpretable and controllable model behavior. We analyze their internal representations, identify residual-stream features associated with each type of refusal, and leverage them to steer model responses at inference. Our contributions are: (1) extract category-specific refusal steering vectors; (2) empirical evidence that our categorical steering reduces over-refusal on ambiguous and benign prompts while preserving refusal on harmful ones across safety benchmarks; and (3) analysis showing that the identified refusal features are distinct, interpretable, and arise from refusal-token fine-tuning.


% Our goal is to assess the effectiveness of token-based refusal as a fine-grained control mechanism to reduce over-refusal without compromising safety.
% We further demonstrate that the identified refusal features emerge from refusal token fine-tuning by comparing them with similarly identified features on the base \textsc{Llama-3 8B}.

% Ensuring the safety and reliability of language models has become a central research priority~\citep{ma2025safetyscalecomprehensivesurvey}, especially their ability to refuse harmful requests while remaining helpful on benign prompt. However, models suffer from \textit{over-refusal}, where harmless or ambiguous prompts are unnecessarily refused, reducing their usability. Several past works have focused on aligning language models with human preferences and expectations, including refusing to answer specific questions based on safety guidelines, through Reinforcement Learning from Human Feedback (RLHF) ~\citep{ouyang2022traininglanguagemodelsfollow} and supervised rule-based methods, such as Constitutional AI ~\citep{bai2022constitutionalaiharmlessnessai}. However, none of these methods address the increasing over-refusal rates. For example, \textsc{Llama-3 8B} exhibits a high over-refusal rate of $\approx 0.69$~\citep{cui2025orbenchoverrefusalbenchmarklarge}. Mechanistic interpretability methods ~\citep{rai2025practicalreviewmechanisticinterpretability} aim to understand the internal computations that drive the behavior of models. Existing approaches for controlling refusal via binary harmful vs. benign signals (e.g., ~\citet{arditi2024refusallanguagemodelsmediated}), overlook fine-grained intent of harmful prompts, fail to isolate features specific to different refusal types, and worsen both interpretability and under-refusal on ambiguous prompts where malicious intents appear benign (~\citep{vonrecum2024notautomaticanalysisrefusal}). To address this, \citet{jain2024refusaltokenssimpleway} fine-tune \textsc{Llama-3 8B} to generate categorical “[refuse]” or “[respond]” tokens, followed by a refusal message or a normal response. There are  five refusal categories that are drawn from~\citet{brahman2024artsayingnocontextual}, enabling for fine-grained refusal categories and a nuanced understanding of refusal behavior.

% In this work, we build on this fine-tuned model to interpret and steer category-specific refusal behavior, aiming to reduce over-refusal while preserving safety and accuracy. By identifying the residual-stream features linked to each refusal category, we can steer fine-grained refusals at inference and better understand the specific residual stream features that cause language models to refuse to answer different categories of harmful prompts. We further show that the discovered features are unique to the refusal token fine-tuned model when compared to the base \textsc{Llama-3 8B}.

% Our contributions are: (i) a simple procedure to extract category-specific sparse steering vectors from residual-stream activations; (ii) empirical evidence that categorical steering reduces over-refusal on ambiguous and benign prompts while maintaining refusal rates on genuinely harmful prompts across safety benchmarks; and (iii) analysis demonstrating that the identified refusal features are distinct and interpretable, emerge from refusal-token fine-tuning, and enable downstream steering without degrading general task performance.


% \section{Related Work}
% \paragraph{Model Refusal} Language models are designed to refuse to respond with unsafe, illegal, harmful, or otherwise malicious responses. Several past studies have focused on aligning language models with human preferences and expectations, including refusing to answer specific questions based on safety guidelines, through Reinforcement Learning from Human Feedback (RLHF) ~\citep{ouyang2022traininglanguagemodelsfollow} and supervised rule-based methods, such as Constitutional AI ~\citep{bai2022constitutionalaiharmlessnessai}.\citet{jain2024refusaltokenssimpleway} fine-tune \textsc{Llama-3 8B} to produce a category-specific refusal or response token at the beginning of model responses based on the content of the user's prompt. However, none of these methods address the increasing over-refusal rates. We employ steering techniques with a fine-grained level of control using specific harmful categories to reduce the over-refusal rate while maintaining the refusal rate on truly harmful prompts.

\paragraph{Understanding and Steering LLM Refusal} Mechanistic interpretability methods ~\citep{rai2025practicalreviewmechanisticinterpretability} aim to understand the internal computations that drive the behavior of models. Previous works have focused on identifying features for simple harmful vs. benign binary tasks, such as ~\citet{arditi2024refusallanguagemodelsmediated} and ~\citet{panickssery2024steeringllama2contrastive}. These binary features are used in steering vectors to steer model behavior towards refusals or responses. However, these methods fail to identify specific features for different types of refusals. We aim to better understand the particular residual stream features that cause language models to refuse to answer different categories of harmful prompts, and to use these features in steering vectors to steer refusal behavior at fine-grained levels.

% \textbf{Sparse Autoencoders.} LLMs tend to encode more features than their dimensions can represent. This creates an issue known as superposition (Elhage et al., 2022), where unrelated concepts become entangled in the model's representation space. When a single neuron responds to multiple unrelated features (polysemanticity), it becomes more difficult to isolate and interpret individual concepts and behaviors, such as refusal. Sparse Autoencoders (SAEs) are neural networks designed to address this issue, mapping model activations into a higher-dimensional space with enforced sparsity via penalties, ensuring only a few neurons fire for each input (Cunningham et al., 2023). This creates a set of disentangled and thus interpretable features that can linearly reconstruct the original activations (Templeton et al., 2024). For example, Yeo et al. (2024) use SAEs to analyze refusal in instruction-tuned LLMs, focusing on simple, binary refusal versus acceptance, but without addressing more nuanced cases, such as ambiguous prompts. 

% Another example of causal intervention is patching. Activation patching ~\citep{zhang2024bestpracticesactivationpatching} involves taking hidden states from a "source" forward pass (for example, when the model complies) and injecting them into a "target" run with different behavior to observe changes in the target output. This replacement helps test the causal role of internal activations ~\citep{vig2020causalmediationanalysisinterpreting}. It has been applied at different levels of granularity, from individual attention heads ~\citep{Elhage_Nanda_2021} to full MLP layers~\citep{geiger2025causalabstractiontheoreticalfoundation}.
% Attribution patching ~\citep{syed2023attributionpatchingoutperformsautomated} is a technique that uses gradients to approximate activation patching more efficiently, ranking components based on their causal effect on a specific output and then selectively patching only the most influential ones. It is more scalable for large models ~\citep{kramár2024atpefficientscalablemethod} and more useful for hypothesis-driven scenarios.
%

% Most of the field of mechanistic interpretability revolves around causal intervention, or testing whether identified individual components are responsible (in a cause-and-effect relationship) for certain behaviors, specifically by intervening in activations or structure. One method of causal intervention is ablation studies ~\citep{meyes2019ablationstudiesartificialneural}, where specific components (e.g., attention heads or even full layers) are disabled to test how they individually contribute to end behavior ~\citep{li2017understandingneuralnetworksrepresentation}. For example, ~\citet{michel2019sixteenheadsreallybetter} disables a large percentage of attention heads used during model training, without retraining the model, and discovers little to no impact on model performance; they describe this process as "pruning down models". Similarly, ~\citet{zhou2018revisitingimportanceindividualunits} turns off individual units or neurons in an image recognition Convolutional Neural Network (CNN), finding a significant drop in class-specific accuracy but not overall vision recognition accuracy. 

\section{Methodology}
% Our objective is to interpret and control fine-grained categories of refusal behavior in language models, to reduce over-refusal while maintaining appropriate refusal on harmful prompts and preserving model performance on benign inputs. We also investigate whether the features associated with each harmful category are unique to the model fine-tuned with refusal tokens, compared to the base \textsc{Llama-3 8B}, through model diffing.

Our methodology involves extracting category-specific features, constructing sparse steering vectors, applying them at inference time, and comparing representational differences with a \textsc{Llama-3 8B Base} model via model diffing. We demonstrate our framework in Figure~\ref{fig:refusal_token_methodology}.

\begin{figure}[H]
    \centering
    \includegraphics[width=\textwidth]{visuals/framework.pdf}
    \caption{Our framework of activation extraction, steering vector computation, and inference-time categorical steering}
    \label{fig:refusal_token_methodology}
\end{figure}

\paragraph{Caching Activations}
Using the fine-tuned refusal token model from~\citet{jain2024refusaltokenssimpleway}, we first extract residual-stream activations at a given layer $l$. Specifically, we target the post-MLP activation for the final token in each input sequence. We experiment with different layers to maximize separation between activations of various categories and to provide the best steering capabilities at inference.

For each of the five harmful categories of prompts and the benign category of prompts, we hook into the model at layer $l$ and extract the residual-stream activation for the last token in each prompt. We then compute mean activations $\mu^l_c$ for each harmful category $c$ and $\mu^l_b$ for the benign category $b$.

% We cache these activations for each harmful category $c$ and use them to calculate the mean activations $\mu^l_c$ for each harmful category. The same is done for the benign category $b$.

\paragraph{Identifying Features and Steering Vectors} \label{sec:compute_features_vectors}
% We apply a methodology similar to 
% those of \textit{Contrastive Activation Addition} ~\citep{panickssery2024steeringllama2contrastive} and 
% \textit{Sparse Activation Steering (SAS)} ~\citep{bayat2025steeringlargelanguagemodel}. 
We apply a similar method of \textit{Sparse Activation Steering (SAS)} ~\citep{bayat2025steeringlargelanguagemodel}, directly to the residual-stream activations of the model rather than a latent autoencoder representation. To construct a steering vector for category $c$, we first threshold the mean activation $\mu^l_c$, retaining only features above a fixed threshold $\tau$, resulting in a filtered mean activation $\tilde{\mu}^l_c$.
% For each of the categories $c$, we begin to construct steering vectors by filtering out features from the mean activations $\mu^l_c$ that are lower than a threshold hyperparameter $\tau$. 
For each harmful category, we compute a steering vector by subtracting the benign category's mean activation from the harmful category's mean activation:
% \begin{equation}
$v_c^l = \tilde{\mu}^l_c - \tilde{\mu}^l_b$.
% \end{equation}
Then, we enforce sparsity in the steering vectors by only keeping the top-$K$ features from each of the category-specific steering vectors, creating $\tilde{v}_c^l$. This is to ensure that steering does not affect general model capabilities.
Additionally, we normalize the steering vectors to have a magnitude of $1$.

\paragraph{Steering Refusal Behavior}

Using the identified steering vectors, we steer model refusal behavior at inference time with the goal of reducing over-refusal while maintaining high refusal rates on genuinely harmful prompts. For each newly generated token, we add the corresponding category-specific steering vector $\tilde{v}_c^l$ to the residual stream activation of the final token at a designated layer $l$. We also apply a strength hyperparameter $\alpha$ to control the magnitude and direction of the intervention:
% \begin{equation}
$\tilde z^{l} = z^{l} + \alpha \ \tilde{v}_c^l$.
% \end{equation}
A positive $\alpha$ amplifies refusal behavior on harmful prompts, while a negative $\alpha$ reduces refusal on benign and ambiguous prompts, thereby reducing over-refusal.

Steering is applied categorically based on the contents of the input prompt. The model selects the most optimal steering vector for application at inference time. This process works by first generating a "[refuse]" or "[respond]" token without any steering, and then using the generated refusal token as a key to map to its corresponding category's specific steering vector $\tilde{v}_c^l$ and steering strength $\alpha$ to steer fine-grained refusal behavior.

\section{Experiments}
% \paragraph{Models} We evaluate using both the base \textsc{Llama-3 8B} and the categorical refusal-token fine-tuned \textsc{Llama-3 8B} model from \citet{jain2024refusaltokenssimpleway}, in which they prepend a category-specific ``[refuse]'' or ``[respond]'' token to model responses. 
% % The refusal token directly precedes a refusal message in the model response. 

% \paragraph{Datasets}  
% To compute steering vectors, we primarily use two splits from the CoCoNot dataset from ~\citet{brahman2024artsayingnocontextual}: (1) \textit{CoCoNot-Orig} with truly harmful prompts and (2) \textit{CoCoNot-Contrast} containing benign but ambiguous prompts used for evaluating over-refusal. Additionally, we evaluate refusal behavior on WildJailbreak ~\citet{jiang2024wildteamingscaleinthewildjailbreaks} and OR-Bench ~\citet{cui2025orbenchoverrefusalbenchmarklarge}, both of which include subsets used for benchmarking over-refusal. We also evaluate general language model performance by measuring accuracy on general benchmarks such as GSM8K ~\citet{cobbe2021trainingverifierssolvemath}, MMLU ~\citet{hendrycks2021measuringmassivemultitasklanguage}, and TruthfulQA ~\citet{lin2022truthfulqameasuringmodelsmimic}.

% \paragraph{Computing and Evaluating Steering Vectors} We choose to hook at the residual-stream activations after the MLP in layer $9$. We set $\tau=10^{-3}$ and $K = 200$ for computing steering vectors. To assess category distinctiveness, we compute pairwise cosine similarities and visualize the vectors via 2D PCA and t-SNE.

% \paragraph{Refusal Rate Evaluation} We evaluate model refusal rates in two ways. The first approach is to use an LLM as a judge, specifically \textsc{Gemini 2.5 Flash} ~\citet{comanici2025gemini25pushingfrontier}, to detect whether model responses contain refusal messages. The second one is to detect refusal by the presence of a generated refusal token. We primarily use the first approach to evaluate \textsc{Llama-3 8B} and the second approach to assess the refusal token fine-tuned model and the steered model.

% % Silhouette, Davies--Bouldin, and Calinski--Harabasz scores

We evaluate four models: (1) the original, non–fine-tuned \textsc{Llama-3 8B Base} as our baseline; (2) the binary refusal-token fine-tuned model from \citet{jain2024refusaltokenssimpleway}, which outputs a generic ``[refuse]'' or ``[respond]'' token; (3) the categorical refusal-token fine-tuned model from \citet{jain2024refusaltokenssimpleway}, which prepends category-specific refusal tokens and is the source of our steering vectors; and (4) our conditionally steered model, which applies categorical steering at inference time.


To compute steering vectors, we use CoCoNot~\citep{brahman2024artsayingnocontextual} with (1) \textit{Orig} for harmful and (2) \textit{Contrast} for ambiguous benign prompts. We evaluate refusal behavior using WildJailbreak~\citep{jiang2024wildteamingscaleinthewildjailbreaks} and OR-Bench~\citep{cui2025orbenchoverrefusalbenchmarklarge}, and assess general model performance on GSM8K~\citep{cobbe2021trainingverifierssolvemath}, MMLU~\citep{hendrycks2021measuringmassivemultitasklanguage}, and TruthfulQA~\citep{lin2022truthfulqameasuringmodelsmimic}.

We evaluate model refusal rates in two ways. The first approach is to use an LLM as a judge, specifically \textsc{Gemini 2.5 Flash} ~\citep{comanici2025gemini25pushingfrontier}, to detect whether model responses contain refusal messages. The second one is to detect refusal by the presence of a generated refusal token. We primarily use the first approach to evaluate \textsc{Llama-3 8B} and the second approach to assess the refusal token fine-tuned model and the steered model.

% To assess refusal behavior, we use two approaches: (1) prompting \textsc{Gemini 2.5 Flash}~\citep{comanici2025gemini25pushingfrontier} as an LLM judge to detect refusals, and (2) identifying generated refusal tokens. The first is used to evaluate the base model, and the second is used for the fine-tuned and steered models.

\section{Results}
\paragraph{Analysis on Category-Specific Steering Vectors and Features}

We steer at the residual stream after the MLP in layer $9$; we selected this site empirically based on preliminary exploration and due to computational constraints.
The computed pairwise cosine similarities between the five category-specific steering vectors at layer 9 have generally low-to-moderate values (Figure~\ref{fig:steering_vector_cos_sim} in Appendix~\ref{app:steering_vector_cos_sim}), indicating partial decorrelation that makes the steering vectors suitable for fine-grained steering control. Notably, the \emph{Incomplete} steering vector is especially decorrelated, indicating that the features for mediating refusal for incomplete requests are unique. We also find that features $4055$ and $290$ are consistently the most active across the steering vectors (Figure~\ref{fig:identified_top_features} in Appendix~\ref{app:identified-features}).

% Geometry diagnostics over the categorical harmful and benign residual-stream activations yield Silhouette Score = $-0.028$, Davies--Bouldin = $10.31$, and Calinski--Harabasz = $0.0$, suggesting weak separability at this site and token position; nevertheless, the cosine structure affords usable steering.

\paragraph{Do Refusal Token Fine-Tuning Induce Emergent Category-Specific Features?}

To validate that our identified refusal features emerge from refusal token fine-tuning, we evaluate the exclusiveness of features from the refusal token fine-tuned model when compared to the base \textsc{Llama-3 8B}. Using model diffing, we compute steering vectors using the same methodology on both models and compute cosine similarities between pairs of steering vectors. Lower cosine similarity values generally indicate that the corresponding features are likely emergent from fine-tuning.

% \begin{wraptable}{r}{0.5\textwidth}
\begin{table}
% \vspace{-1.2em}
\centering
\caption{Model diffing cosine similarities.}
% \resizebox{0.95\linewidth}{!}{
\begin{tabular}{lc}
\toprule
Category & Cosine Sim \\
\midrule
Requests with safety concern & 0.336 \\
Humanizing requests & 0.317 \\
Incomplete requests & 0.651 \\
Unsupported requests & 0.333 \\
Indeterminate requests & 0.334 \\
\bottomrule
\end{tabular}
% }
\label{tab:model-diffing-cosine-sims}
% \vspace{-0.9em}
\end{table}

Across most categories, cross-model similarities are low (0.317 -- 0.336), while \emph{Incomplete} shows a higher alignment (0.651) (Table~\ref{tab:model-diffing-cosine-sims}), suggesting partial reuse of base model features in that case. Overall, this pattern of low-to-moderate similarity supports the hypothesis that refusal-token fine-tuning induces novel, category-specific, refusal-mediating features.

\paragraph{Can Categorical Steering Reduce Over-Refusal Without Compromising Safety?}

We evaluate refusal behavior and safety performance across \textsc{Llama-3 8B Base}, the binary and categorical refusal-token–fine-tuned model, and our categorically steered model. On all three benchmarks, we see that steering significantly reduces over-refusal on ambiguous and benign prompts while preserving the refusal rate on truly harmful requests. Specifically, on CoCoNot Contrast (benign but ambiguous prompts), over-refusal drops from 0.106 to 0.0 with steering, while refusal on CoCoNot Orig (harmful prompts) increases from 0.666 to 0.716 (Table~\ref{tab:safety-results}). Similar trends hold on WildJailbreak and OR-Bench.

% \paragraph{CoCoNot} We evaluate refusal rates on CoCoNot Orig (harmful) and CoCoNot Contrast (ambiguous, benign).

\begin{table}[H]
\centering
\caption{Refusal rates across safety benchmarks, grouped by benign vs. harmful.}
\resizebox{0.9\linewidth}{!}{
\begin{tabular}{lcccc}
\toprule
Dataset & \textsc{Llama-3 8B Base} & \shortstack{Binary\\Tokens FT} & \shortstack{Categorical\\Tokens FT} & \textbf{\shortstack{Categorically\\Steered (Ours)}} \\
\midrule
\multicolumn{5}{l}{\emph{Benign prompts (lower is better)}} \\
\addlinespace
CoCoNot Contrast (Benign) & 0.045 & 0.124 & 0.106 & \textbf{0.0} \\
WildJailbreak Adversarial Benign & 0.148 & 0.138 & 0.086 & \textbf{0.0} \\
OR-Bench Hard (Benign) & 0.180 & 0.497 & 0.388 & \textbf{0.010} \\
\midrule
\multicolumn{5}{l}{\emph{Harmful prompts (higher is better)}} \\
\addlinespace
CoCoNot Orig (Harmful) & 0.198 & 0.715 & 0.666 & \textbf{0.716} \\
WildJailbreak Adversarial Harmful & \textbf{0.565} & 0.245 & 0.222 & 0.225 \\
OR-Bench Toxic (Harmful) & 0.214 & 0.685 & 0.785 & \textbf{0.789} \\
\bottomrule
\end{tabular}
}
\label{tab:safety-results}
\end{table}

% Additionally, we report results for \textsc{Zephyr-Llama3 8B SFT}, which reduces benign over-refusal relative to the FT baseline but remains above our steered model (CoCoNot-Contrast 0.098; WJ-Benign 0.171; OR-Bench Hard 0.497), while maintaining high refusal on harmful prompts (OR-Bench Toxic 0.904; WJ-Harmful 0.389).

% On CoCoNot Contrast, the over-refusal rate drops (0.106,$\to$\,0.0) on ambiguous prompts (Tab.~\ref{tab:coconot}). On CoCoNot Orig, the refusal rate increases (0.666,$\to$\,0.716) on harmful prompts (Tab.~\ref{tab:coconot}).

% Results for WildJailbreak and OR-Bench are included in Table~\ref{tab:safety-results}.

\paragraph{Does Categorical Steering Preserve General Model Performance?}

% \paragraph{General Performance Metrics}
As shown in Table~\ref{tab:general-performance}, the steered model achieves identical accuracy to the refusal-token–fine-tuned model across all three general benchmarks: MMLU, GSM8k, and TruthfulQA.

\begin{table}[H]
\centering
\caption{General Performance Metrics .}
\resizebox{0.9\linewidth}{!}{
\begin{tabular}{llccc}
\toprule
Dataset & \textsc{Llama-3 8B Base} & \shortstack{Binary\\Tokens FT} & \shortstack{Categorical\\Tokens FT} & \textbf{\shortstack{Categorically\\Steered (Ours)}} \\
\midrule
MMLU & $0.6206\pm0.0038$ & $0.5861\pm0.0039$ & $0.5887\pm0.0039$ & $0.5887\pm0.0039$ \\
GSM8k & $0.5057\pm0.0138$ & $0.4496\pm0.0137$ & $0.4534\pm0.0137$ & $0.4534\pm0.0137$ \\
% GSM8k Strict Match & $0.5034\,\pm\,0.0138$ & $0.4526\,\pm\,0.0137$ & $0.4526\,\pm\,0.0137$ \\
TruthfulQA MC & $0.2717\pm0.0156$ & $0.3158\pm0.0163$ & $0.3158\pm0.0163$ & $0.3158\pm0.0163$ \\
% TruthfulQA MC 2 & $0.4399\,\pm\,0.0139$ & $0.4568\,\pm\,0.0148$ & $0.4811\,\pm\,0.0150$ & $0.4811\,\pm\,0.0150$ \\
\bottomrule
\end{tabular}
}
\label{tab:general-performance}

\end{table}

% On general benchmarks, \textsc{Zephyr-Llama3 8B SFT} is comparable to the FT and steered models (e.g., MMLU 0.5916, GSM8k 0.4594, TruthfulQA MC1/MC2 0.3011/0.4568; Table~\ref{tab:general-performance}).

% \section{Discussion}
% % - \textbf{Category geometry} Within-model category directions are low-to-moderately correlated and show weak cluster quality (negative Silhouette), yet remain usable for targeted control (Sec. ~\ref {fig:steering_vector_cos_sim}).

% These signals support the hypothesis that categorical refusal tokens shape the representation space in a directionally meaningful way that is only partly aligned with pre-existing base-model features.

\subsection{Limitations and Future Work}

- \textbf{Evaluation fidelity} LLM-as-judge metrics depend on the judge’s criteria and calibration. While convenient and widely used, they may not perfectly capture user utility.
- \textbf{Coverage} We focus on a single steering site (layer 9, resid\_post) and five refusal categories. Other sites/layers and broader category taxonomies could strengthen generality.
- \textbf{Metrics availability} Some base-model token-level metrics are unavailable (N/A), limiting direct like-for-like comparisons.
- \textbf{Open analyses} We did not include inference-time steering ablations or SAE-based steering in the final tables. Integrating these and reporting confidence intervals for all tasks are priorities for follow-up.

\paragraph{Outlook} The results indicate a promising path: targeted safety gains with emerging robustness to adversarially benign prompts. Future work should pursue multi-objective fine-tuning to mitigate capability regressions, richer mechanistic probes (e.g., causal tracing), and calibrated control policies that adaptively modulate category-specific directions at inference time.


\section{Conclusion}
We demonstrated that categorical refusal tokens induce sparsifiable fine-grained directions in the residual stream, enabling categorical steering. Specifically, over-refusal drops to near zero on benign and ambiguous prompts, while refusal rates on harmful inputs are maintained, and general language model capabilities remain unchanged. Cross-model comparisons suggest that these directions emerge primarily from refusal-token fine-tuning rather than pre-existing base-model features. Building on our findings, we are exploring more advanced methodologies to both enhance safety-performance trade-offs and deepen understanding of the underlying mechanisms. Although this is ongoing work, our preliminary results suggest that steering with categorical refusal tokens is a promising path to balance safety and usability in language models. 


% We showed that categorical refusal tokens induce fine‑grained, sparsifiable directions in the residual stream that support categorical steering: over‑refusal drops to near zero on ambiguous prompts while refusal on harmful inputs is maintained, and general capabilities remain unchanged. Cross‑model comparisons suggest these directions largely emerge from refusal‑token fine‑tuning rather than pre‑existing base‑model features.

% \paragraph{Limitations} Our analysis centers on a single model family and one steering site (layer 9, resid\_post) with fixed thresholds and top‑$K$ sparsification; alternate layers/sites and hyperparameters may yield different trade‑offs. LLM‑as‑judge evaluation introduces calibration bias, and some token‑level metrics are unavailable for strict apples‑to‑apples comparisons. We do not report full ablations (e.g., SAE‑based steering, causal tracing) or uncertainty estimates for all metrics.

% \paragraph{Future Work} Extend to multi‑site, adaptive control policies that select category‑specific strengths on the fly; integrate causal probes and sparsifiers to isolate mechanistic pathways; pursue multi‑objective fine‑tuning to recover small capability regressions; and broaden category taxonomies and datasets to test robustness under distribution shift and adversarial prompting.



\bibliographystyle{plainnat}
\bibliography{neurips_2025}

\clearpage
\appendix
\section{Additional Experiment Details}\label{app:exp}
\subsection{Pairwise Steering Vector Cosine Similarities}\label{app:steering_vector_cos_sim}

\begin{figure}[H]
\centering
\begin{minipage}{0.50\textwidth}
    \includegraphics[width=\linewidth]{visuals/steering_vector_cos_sim - 9 - resid_post.png}
    \caption{Cosine similarities between steering vectors.}
    \label{fig:steering_vector_cos_sim}
\end{minipage}\hfill
\end{figure}

\subsection{Identified Features}\label{app:identified-features}

\begin{figure}[H]
\centering
\includegraphics[width=\linewidth]{visuals/steering_vectors_grouped.png}
\caption{Absolute feature values for features $4055$, $290$, $1039$, $682$, and $87$.}
\label{fig:identified_top_features}
\end{figure}

Examining the top values of the identified features, some shared high-weight features recur across categories, notably indices $4055$, $290$, $682$ (and $1039$), while other indices are more category-specific (e.g., $3881$ and $1421$ for Incomplete). Figure~\ref {fig:identified_top_features} visualizes the values for five representative feature indices across all five harmful categories.

% WildJailbreak and OR-Bench results moved into the main results table

\end{document}